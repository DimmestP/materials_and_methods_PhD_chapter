%
% Sample SBC book chapter
%
% This is a public-domain file.
%
% Charset: ISO8859-1 (latin-1) áéíóúç
%
\documentclass{SBCbookchapter}
\usepackage[utf8]{inputenc}
\usepackage[T1]{fontenc}
\usepackage[brazil,english]{babel}
\usepackage{graphicx}
\usepackage{array}

\setcounter{chapter}{2}

\author{Sam Haynes}
\title{Materials and Methods}

\begin{document}
\maketitle

\begin{abstract}
This chapter introduces open-source software practices, community development and the pedagogy of software documentation to promote the development of long-term, high impact research software. Focusing on the issue of the standardisation of qPCR analysis, the chapter contrasts the R package tidyqpcr, developed by the author, to other current software available. This use case highlights how quality software supported by comprehensive documentation can improve the quality of the entire experimental assay.


\end{abstract}

\section{Software Development}

\section{RNA-seq Analysis Pipelines}

\section{Linear Regression}

\section{Gaussian Processes}
Gaussian processes are a Bayesian method developed for analysing time series data. Time series data is structured differently to other continuous data types as it often has strong autocorrelation between neighbouring points and repeating cycles. Standard linear regression models often struggle to model this behaviour (although circular patterns can be learnt from finite linear combination of sin/cos functions). They are the projection of Gaussian distribution into infinite dimensions. By defining the covariance function, which relates the value of each point to every other point, can a various attributes to the fitted model. For example, if the fitted line must be continuous and differentiable then the covariance functions must be similarly defined. Similar to a finite multi-dimensional Gaussian it is fully defined by two parameters:  the mean function and the covariance function. As it is a Bayesian method it also create an error at in the expected value at any point. The covariance function defines how smooth the function mapping neighboring points are. You can use a number of well characterised functions as the covariance function or you can use a neural network to learn complex connections. 

The omniplate python package developed by the Swain lab uses a Gaussian process with an infinitely differentiable covariance functions. The enables the time series protein florescence data to be model, then doubly differentiated to find the point of max increase in florescence. It is important to use a similar time point to compare all the different constructs at and the time of max growth rate is appropriate. Partly because it is simple to calculate whilst making sure all the constructs are being compared at a similar point in the growth cycle (mid exponential). It also important to account for auto-fluorescence and the size of cells in various media in order to remove confounding contributions to sample fluorescence.

\section{Bayesian Statistical Models}

\bibliographystyle{apalike}
\bibliography{sbc-template}

\end{document}